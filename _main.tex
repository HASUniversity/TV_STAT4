\documentclass[]{book}
\usepackage{lmodern}
\usepackage{amssymb,amsmath}
\usepackage{ifxetex,ifluatex}
\usepackage{fixltx2e} % provides \textsubscript
\ifnum 0\ifxetex 1\fi\ifluatex 1\fi=0 % if pdftex
  \usepackage[T1]{fontenc}
  \usepackage[utf8]{inputenc}
\else % if luatex or xelatex
  \ifxetex
    \usepackage{mathspec}
  \else
    \usepackage{fontspec}
  \fi
  \defaultfontfeatures{Ligatures=TeX,Scale=MatchLowercase}
\fi
% use upquote if available, for straight quotes in verbatim environments
\IfFileExists{upquote.sty}{\usepackage{upquote}}{}
% use microtype if available
\IfFileExists{microtype.sty}{%
\usepackage{microtype}
\UseMicrotypeSet[protrusion]{basicmath} % disable protrusion for tt fonts
}{}
\usepackage[margin=1in]{geometry}
\usepackage{hyperref}
\hypersetup{unicode=true,
            pdftitle={Stat1: statistische modellen},
            pdfauthor={Mark Smits},
            pdfborder={0 0 0},
            breaklinks=true}
\urlstyle{same}  % don't use monospace font for urls
\usepackage{longtable,booktabs}
\usepackage{graphicx,grffile}
\makeatletter
\def\maxwidth{\ifdim\Gin@nat@width>\linewidth\linewidth\else\Gin@nat@width\fi}
\def\maxheight{\ifdim\Gin@nat@height>\textheight\textheight\else\Gin@nat@height\fi}
\makeatother
% Scale images if necessary, so that they will not overflow the page
% margins by default, and it is still possible to overwrite the defaults
% using explicit options in \includegraphics[width, height, ...]{}
\setkeys{Gin}{width=\maxwidth,height=\maxheight,keepaspectratio}
\IfFileExists{parskip.sty}{%
\usepackage{parskip}
}{% else
\setlength{\parindent}{0pt}
\setlength{\parskip}{6pt plus 2pt minus 1pt}
}
\setlength{\emergencystretch}{3em}  % prevent overfull lines
\providecommand{\tightlist}{%
  \setlength{\itemsep}{0pt}\setlength{\parskip}{0pt}}
\setcounter{secnumdepth}{5}
% Redefines (sub)paragraphs to behave more like sections
\ifx\paragraph\undefined\else
\let\oldparagraph\paragraph
\renewcommand{\paragraph}[1]{\oldparagraph{#1}\mbox{}}
\fi
\ifx\subparagraph\undefined\else
\let\oldsubparagraph\subparagraph
\renewcommand{\subparagraph}[1]{\oldsubparagraph{#1}\mbox{}}
\fi

%%% Use protect on footnotes to avoid problems with footnotes in titles
\let\rmarkdownfootnote\footnote%
\def\footnote{\protect\rmarkdownfootnote}

%%% Change title format to be more compact
\usepackage{titling}

% Create subtitle command for use in maketitle
\newcommand{\subtitle}[1]{
  \posttitle{
    \begin{center}\large#1\end{center}
    }
}

\setlength{\droptitle}{-2em}

  \title{Stat1: statistische modellen}
    \pretitle{\vspace{\droptitle}\centering\huge}
  \posttitle{\par}
    \author{Mark Smits}
    \preauthor{\centering\large\emph}
  \postauthor{\par}
      \predate{\centering\large\emph}
  \postdate{\par}
    \date{2019-04-24}

\usepackage{booktabs}

\usepackage{amsthm}
\newtheorem{theorem}{Theorem}[chapter]
\newtheorem{lemma}{Lemma}[chapter]
\newtheorem{corollary}{Corollary}[chapter]
\newtheorem{proposition}{Proposition}[chapter]
\newtheorem{conjecture}{Conjecture}[chapter]
\theoremstyle{definition}
\newtheorem{definition}{Definition}[chapter]
\theoremstyle{definition}
\newtheorem{example}{Example}[chapter]
\theoremstyle{definition}
\newtheorem{exercise}{Exercise}[chapter]
\theoremstyle{remark}
\newtheorem*{remark}{Remark}
\newtheorem*{solution}{Solution}
\let\BeginKnitrBlock\begin \let\EndKnitrBlock\end
\begin{document}
\maketitle

{
\setcounter{tocdepth}{1}
\tableofcontents
}
\chapter*{Voorwoord}\label{voorwoord}
\addcontentsline{toc}{chapter}{Voorwoord}

Afgelopen blok hebben jullie al aardig wat data verzameld met practica,
en de meeste ook al met hun bioxperience-project. In dit blok leren
jullie om {[}data verzameld in practica en bioxperience{]}{[}dit blok
gaan we di{]}

\section*{Opzet}\label{opzet}
\addcontentsline{toc}{section}{Opzet}

\chapter{Statistische vragen bij
grafieken}\label{statistische-vragen-bij-grafieken}

\BeginKnitrBlock{ABD}
Lees Chapter 1 (\emph{Statistics and samples}):

\begin{itemize}
\tightlist
\item
  Helemaal
\end{itemize}

Lees Chapter 2 (\emph{Displaying data}):

\begin{itemize}
\tightlist
\item
  Helemaal

  \EndKnitrBlock{ABD}
\end{itemize}

\section{Soorten grafieken}\label{soorten-grafieken}

Een grafiek is dé manier om data te presenteren. Het laat in één
oogopslag zien wat voor patronen er in de data zitten. Verschillende
soorten data vragen om verschillende soorten grafieken.

Een aantal mogelijke soorten grafieken: * Boxplot * Spreidingsdiagram *
Staafdiagram * Taartdiagram/mozaiekplot

\BeginKnitrBlock{exercise}
\protect\hypertarget{exr:unnamed-chunk-2}{}{\label{exr:unnamed-chunk-2}
}Soorten figuren

\begin{itemize}
\tightlist
\item
  Zoek in de literatuur (bijvoorbeeld die je verzameld hebt voor
  bioxperience) naar voorbeelden voor ieder van bovenstaande typen
  figuren, en kopieer ze naar je portfolio.
\end{itemize}
\EndKnitrBlock{exercise}

Data op de juiste manier presenteren is de eerste stap in het analyseren
van je data.

\section{Soorten data}\label{soorten-data}

\BeginKnitrBlock{exercise}
\protect\hypertarget{exr:unnamed-chunk-3}{}{\label{exr:unnamed-chunk-3}
}Soorten data

Data kan verdeeld worden in drie niveau's.

\begin{itemize}
\tightlist
\item
  Zoek in \emph{Analysis of Biological Data} op welke niveau's dat zijn,
  noteer ze in je portfolio.
\item
  Benoem het niveau van de data in de figuren die je in de vorige
  opdracht hebt verzameld.

  \EndKnitrBlock{exercise}
\end{itemize}

\section{Statistische vragen}\label{statistische-vragen}

Achter ieder onderzoek, en dus ook achter iedere dataset, zitten een of
meerdere onderzoeksvragen. De kunst is om een figuur zo op te zetten dat
de vraag daarmee beantwoord kan worden. Een onderzoeksvraag wordt
daarmee automatisch ook een statistische vraag omdat je met data uit een
steekproef werkt. Daarmee heb je te maken met onzekerheid.

\BeginKnitrBlock{exercise}
\protect\hypertarget{exr:unnamed-chunk-4}{}{\label{exr:unnamed-chunk-4}
}Onderzoeksvragen

\begin{itemize}
\tightlist
\item
  Zoek uit welke onderzoeksvragen horen bij de figuren die je hebt
  uitgezocht in de vorige opdrachten.
\item
  Zoek uit (tip: check materiaal en methode) welke statistische toets
  gebruikt is om de onderzoeksvraag te beantwoorden.
\item
  Wat is de conclusie van de statistische toets?
\item
  Noteer alle antwoorden in je portfolio.
\end{itemize}
\EndKnitrBlock{exercise}

\chapter{Data beschrijven}\label{data-beschrijven}

\BeginKnitrBlock{ABD}
Lees: Chapter 3 (\emph{Arithmetic mean and standard deviation}):

\begin{itemize}
\tightlist
\item
  Paragraph 3.1 (\emph{Arithmetic mean and standard deviation})
\end{itemize}

Lees: Chapter 4 (\emph{Estimating with uncertainty}):

\begin{itemize}
\tightlist
\item
  Helemaal

  \EndKnitrBlock{ABD}
\end{itemize}

Sommige figuren laten de ruwe data zien. Een goed voorbeeld is een
spreidingsdiagram. Andere figuren geven alleen het gemiddelde en de
spreiding, zoals in de meeste staafdiagrammen.

Met `verklarende statistiek' probeer je de variatie in je data te
verklaren. Dat doe je aan de hand van een statistisch model.

\BeginKnitrBlock{exercise}
\protect\hypertarget{exr:unnamed-chunk-6}{}{\label{exr:unnamed-chunk-6} }
\EndKnitrBlock{exercise}

\section{Verklarende variabelen}\label{verklarende-variabelen}

\section{Overschrijdingskans}\label{overschrijdingskans}

\section{Welke toetsen voor welke
data?}\label{welke-toetsen-voor-welke-data}

\subsection{2 groepen}\label{groepen}

\subsection{3 groepen}\label{groepen-1}

\subsection{Correlatie/Regressie}\label{correlatieregressie}

\chapter{Power en steekproefgrootte}\label{power-en-steekproefgrootte}

\BeginKnitrBlock{ABD}
Lees Chapter 6 (\emph{Hypothesis testing}):

\begin{itemize}
\tightlist
\item
  Helemaal

  \EndKnitrBlock{ABD}
\end{itemize}

\section{Foute conclusies}\label{foute-conclusies}

\section{Power}\label{power}

\section{Minimale steekproefgrootte}\label{minimale-steekproefgrootte}

\section{Simuleren}\label{simuleren}

\chapter{t-toets voor één steekproef}\label{t-toets-voor-een-steekproef}

\BeginKnitrBlock{ABD}
Lees Chapter 11 (\emph{Inference for a normal population}):

\begin{itemize}
\tightlist
\item
  Paragraph 11.3 (\emph{The one-sample t-test})

  \EndKnitrBlock{ABD}
\end{itemize}

\chapter{Gepaarde t-toets}\label{gepaarde-t-toets}

\BeginKnitrBlock{ABD}
Lees Chapter 12 (\emph{Comparing two means}):

\begin{itemize}
\tightlist
\item
  Paragraph 12.1 (\emph{Paired sample versus two independent samples})
\item
  Paragraph 12.2 (\emph{Paired comparison of means})
\item
  Paragraph 12.3 (\emph{TWo-sample comparison of means})

  \EndKnitrBlock{ABD}
\end{itemize}

\chapter{Voorwaarden t-toetsen:}\label{voorwaarden-t-toetsen}

\BeginKnitrBlock{ABD}
Lees Chapter 12 (\emph{Comparing two means}):

\begin{itemize}
\tightlist
\item
  Paragraph 12.5 (\emph{The fallacy of indirect comparison})
\end{itemize}

Lees Chapter 13 (\emph{Handling violations of assumptions}):

\begin{itemize}
\tightlist
\item
  Paragraph 13.1 (\emph{Detecting deviations from normality})
\item
  Paragraph 13.2 (\emph{When to ignore violations of assumptions})
\item
  Paragraph 13.3 (\emph{Data transformations})

  \EndKnitrBlock{ABD}
\end{itemize}

\section{Onafhankelijke waarnemingen}\label{onafhankelijke-waarnemingen}

\section{Normaliteit als basis voor statistische
toetsen}\label{normaliteit-als-basis-voor-statistische-toetsen}

\begin{itemize}
\tightlist
\item
  teruggrijpen op statistisch model
\item
  studentized residuals normaal verdeeld
\end{itemize}

\section{Wanneer normaal verdeeld?}\label{wanneer-normaal-verdeeld}

\subsection{Skewness}\label{skewness}

\subsection{Kurtosis}\label{kurtosis}

\subsection{Statistische toets op
normaliteit}\label{statistische-toets-op-normaliteit}

\subsection{Visueel via QQ-plots of
histogrammen}\label{visueel-via-qq-plots-of-histogrammen}

\section{Transformatie}\label{transformatie}


\end{document}
